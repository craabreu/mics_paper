%\documentclass[journal=jctcce,manuscript=article]{achemso}
\documentclass[journal=jctcce,manuscript=article,layout=twocolumn]{achemso}

% TODO: comment below before submitting
\let\titlefont\undefined
\makeatletter
\let\l@addto@macro\relax
\makeatother
\usepackage[fontsize=11pt]{scrextend}
% TODO: comment above before submitting

\usepackage{amsmath}
\usepackage{amssymb}
\usepackage[T1]{fontenc}
\usepackage[inline]{enumitem}
\usepackage{hyperref}
%
\usepackage[table]{xcolor}
\definecolor{lightgray}{gray}{0.9}
%
\usepackage{array}
\newcolumntype{L}{>{$}l<{$}}
\newcolumntype{C}{>{$}c<{$}}
\newcolumntype{R}{>{$}r<{$}}

\newcommand{\mt}[1]{\boldsymbol{\mathbf{#1}}}   % matrix symbol
\newcommand{\vt}[1]{\boldsymbol{\mathbf{#1}}}   % vector symbol
\newcommand{\tr}[1]{#1^\text{t}}                % transposition
\newcommand{\diff}[2]{\frac{\partial #1}{\partial #2}} % derivative
\newcommand{\avg}[1]{\overline{#1}}             % average

%\listfiles

\author{Charlles R. A. Abreu}
\email{abreu@eq.ufrj.br}
\affiliation{Chemical Engineering Department, Escola de Quimica, Universidade Federal do Rio de Janeiro, Rio de Janeiro, RJ 21941-909, Brazil}

\title{Free Energy Computation and Property Reweighting Using Multiple Time-Correlated Datasets}

%\abbreviations{i.i.d., MBAR, MICS}
%\keywords{Free Energy Computation, Reweighting, Multistate, Uncertainty Estimation}

\begin{document}

%\begin{tocentry}
%Graphical Abstract
%\end{tocentry}

%\tableofcontents

\begin{abstract}
Abstract.
\end{abstract}

\section{Introduction}
\label{sec:introduction}

In order to quantify the relative propensity of a system to be found in either of two given equilibrium states, we must be able to compute the difference in their free energies. Besides, the rate at which a non-equilibrium process will occur depends on the free-energy profiles of realizable paths connecting the end states. For these and other reasons, free energy calculation is a task of paramount importance in Computational Chemistry and related areas \cite{Chipot_2007, Christ_2010, Hansen_2014}. Among the applications in which the knowledge of free energy changes plays a central role, we can cite protein-ligand binding \cite{Chodera_2011, Abel_2017, Abel_2017_2, Cournia_2017, Mobley_2017} and other host-guest interactions \cite{X}, protein folding \cite{Perez_2016, X}, adsorption and structural transitions in metal-organic frameworks \cite{Coudert_2008, Bousquet_2012, Ghysels_2013, Demuynck_2017}, and clathrate hydrate formation \cite{X}, to name just a few.

Paraphrasing \citeauthor{Christ_2010} \cite{Christ_2010}, any free energy calculation endeavor must meet at least three requirements: \begin{enumerate*}[label = \arabic*)] \item a proper model for representing the geometry and energetics of the system of interest, \item a strategy for sampling the system's configurational space according to some probability distribution, and \item a method for analyzing the collected samples, which should always include estimating the uncertainties of the computed free energies.\end{enumerate*} It is worth adding that some methods are designed to fulfill the two latter ones simultaneously.

The present paper is concerned with the third requirement above. In this context, Bennett's Acceptance Ratio (BAR) method \cite{Bennett_1976} has survived many challenges over time and stood as the best method for direct computation of the free energy difference between two equilibrium states sampled separately \cite{Lu_2003, Shirts_2005}. An important advance after Bennett's work, put forward by \citeauthor{Shirts_2008} \cite{Shirts_2008}, was presented as an extension of BAR to the joint analysis of multiple states and was thus named MBAR (M for multistate). It supersedes BAR when the two states of interest are poorly connected and must be bridged together by intermediate ones, in which case BAR can only be applied in stages.

Besides the (relative) free energies of all sampled states, MBAR is able to interpolate properties and, just as importantly, provide their uncertainties at a continuum of unsampled states. Such a powerful tool is known as perturbation \cite{Zwanzig_1954} or reweighting \cite{McDonald_1967, McDonald_1969}, depending on whether the computed quantity is a free energy or a plain ensemble average. The use of MBAR to build continuous curves has, for instance, allowed us to significantly enhance our analysis of the behavior of some supercritical fluids \cite{Aimoli_2014, Aimoli_2014_2, Nichele_2018}.

MBAR can also be viewed as a limiting case of the Weighted Histogram Analysis Method (WHAM) \cite{Kumar_1992}, with the benefits of dispensing with binning \cite{Tan_2012} and being born equipped with a proper uncertainty estimator \cite{Shirts_2008}. Indeed, it is the second feature what distinguishes MBAR from earlier binless versions of WHAM \cite{Bartels_2000, Souaille_2001}. The method descends from the maximum likelihood approach of \citeauthor{Kong_2003} \cite{Kong_2003}, which is in turn similar to other methods published in the Statistics literature.\cite{Vardi_1985, Gill_1988, Geyer_1994, Lindsay_1995, Meng_1996}. A reliable uncertainty estimation depends on the proof of a specific Central Limit Theorem (CLT), which in the case of MBAR requires the samples to be absent of autocorrelation. This very often enforces one to discard many blocks of correlated data before applying the method, in a process known as subsampling. Even though most of the statistically meaningful content of the sample remains after subsampling, the discarded data could possibly have a non-negligible contribution to reducing uncertainties and producing smoother profiles via reweighting.

Our aim here is to develop a method that shares the same purposes as MBAR, but whose uncertainty estimator can cope with autocorrelations. The method relies on recent developments and CLT proofs from the Statistics literature\cite{Flegal_2010, Buta_2010, Buta_2011, Doss_2014, Vats_2015, Tan_2015, Roy_2018} and is based on Geyer's Reverse Logistic Regression and Reweighting Mixtures \cite{Geyer_1994}. We refer to the method as MICS, standing for Mixture of Independently Collected Samples. In contrast to the index-based notation employed in these references, we resort to differentiation in matrix notation (especially a matrix-based chain rule) and thus manage to derive remarkably compact expressions.

We start our exposition with a contextualizing preamble, aiming at reviewing some basic concepts and establishing our train of thought. In this part, we offer an interpretation of MBAR as being equivalent to the analysis of an expanded-ensemble simulation \cite{Lyubartsev_1992} whose importance weights are unknown to the analyzer. This view complements a recent publication by \citeauthor{Shirts_2017} \cite{Shirts_2017} and hopefully contributes to the assimilation of both MBAR and MICS. After that, we present the new method itself, including its maximum-likelihood estimator for the free energies of the sampled states and its machinery for performing free-energy perturbation and reweighting. In all cases,  uncertainty estimators are properly derived. Finally, we test MICS's performance in real-world applications, showing that it can lead to better estimates when compared to MBAR.

Let us end this introduction by briefly contrasting MICS to other generalizations of MBAR. Although the assumption of independence is relaxed in MICS, each sample must still consist of identically distributed configurations and thus reproduce an equilibrium state. Besides, the sequence of configurations must depict a Markovian walk with, at least, polynomial ergodicity \cite{Roy_2018}. In an interesting class of recently proposed methods \cite{Mey_2014, Wu_2014, Rosta_2015, Wu_2016}, such a global equilibrium requirement is suppressed at the cost of building a multiensemble version \cite{Wu_2016} of a Markov State Model (MSM) \cite{Pande_2010, Husic_2018}. This involves stratifying all samples into bins (such as in WHAM) and then assuming that \begin{enumerate*}[label = \arabic*)] \item configurations are identically distributed within each bin (local equilibrium) and \item each sample depicts a Markovian walk through bins.\end{enumerate*} Hence, the need for subsampling remains, but this time due to correlations amongst bin-to-bin transitions. Furthermore, no expressions for estimating uncertainties have thus far been devised for these MSM-based methods. Clearly, the purpose and the development strategy adopted for MICS are distinct.

\section{Preamble}

\subsection{Importance Sampling, Uncertainty Analysis, and the Delta Method}
\label{sec:definitions}

Let $x$ represent the set of coordinates of a system which can be found at different equilibrium states. Each state $i$ is a statistical ensemble whose probability density is given by
\begin{equation}
\label{eq:state_prob_density}
\rho_i(x) = \frac{1}{Z_i} e^{-u_i(x)},
\end{equation}
where $u_i(x)$ is a reduced potential \cite{Shirts_2008, Chodera_2011_2} with known functional form, while $Z_i = \int e^{-u_i(x)}dx$ is the configurational integral of the system at state $i$. The ensemble average of a property $A(x)$ at state $i$ is defined as
\begin{equation}
\label{eq:ensemble average}
\langle A \rangle_i = \int A(x)\rho_i(x)dx.
\end{equation}

Assume that a sample of configurations $S_i = \{x_{i,k}\}_{k=1}^{n_i}$ has been obtained via importance sampling \cite{Allen_1987} (e.g. MC or MD) at state $i$. In this case, we can estimate $\langle A \rangle_i$ by a simple arithmetic average, that is,
\begin{equation}
\label{eq:average estimator}
\avg A_i = \frac{1}{n_i} \sum_{k=1}^{n_i} A(x_{i,k}).
\end{equation}

This estimator naturally extends to multivalued properties, such as vectors and matrices. It is valid whether the configurations in $S_i$ are independent and identically distributed (i.i.d.) or form a correlated time series. The difference resides in how to estimate the uncertainty of an average $\avg A_i$ or, more generally, of a function $h(\avg {\vt y}_i)$, where $\vt y(x)$ is a vector of configurational properties $\tr{[A(x) \; B(x) \; \cdots]}$. Let us define $\hat{\mt \Sigma}_{\avg{\vt y}_i}$ as the matrix of estimated asymptotic covariances between the entries of $\avg{\vt y}_i$. From the central limit theorem (CLT) applied to an i.i.d. sample, such matrix can be obtained by
\begin{equation}
\label{eq:asymptotic covariance IID}
\hat{\mt \Sigma}^\ast_{\avg{\vt y}_i} = \frac{\sum\limits_{k=1}^{n_i} \left[\vt y(x_{i,k}) - \avg{\vt y}_i\right] \tr{\left[\vt y(x_{i,k}) - \avg{\vt y}_i\right]}}{n_i(n_i - 1)}.
\end{equation}

This is not true if the sample drawn at state $i$ is a time-correlated series. Several methods exist for estimating asymptotic variances in this case, such as batching, spectral analysis, and regenerative simulations \cite{Geyer_1992, Alexopoulos_2006, Flegal_2010, Doss_2014}. Extensions of these methods to the computation of covariance matrices have also been validated \cite{Vats_2015, Vats_2018}. Spectral methods usually perform better than batching \cite{Flegal_2010}, but they are more computationally demanding, especially in the multivariate case \cite{Vats_2015}. In this context, the overlapping batch-mean (OBM) estimator \cite{Meketon_1984} is particularly appealing for being asymptotically equivalent to a spectral method while still being very efficient and easy to implement. For a sample $S_i$, it consists in defining $n_i-b_i+1$ overlapping blocks of size $b_i$. The average of $\vt y(x)$ is ${\avg{\vt y}}^b_{i,1} = \frac{1}{b_i} \sum_{k=1}^{b_i} \vt y(x_{i,k})$ for the first block and, for the remaining ones, they can be computed recursively by
\begin{equation*}
{\avg{\vt y}}^b_{i,j+1} = {\avg{\vt y}}^b_{i,j} + \frac{\vt y(x_{i,j+b_i}) - \vt y(x_{i,j})}{b_i}.
\end{equation*}

Then, the OBM estimator for the matrix of asymptotic covariances among the entries of $\avg{\vt y}_i$ is computed by \cite{Meketon_1984}
\begin{equation}
\label{eq:obm asymptotic covariance}
\hat{\mt \Sigma}_{\avg{\vt y}_i} = \frac{b_i \sum\limits_{j=1}^{n_i - b_i + 1} ({\avg{\vt y}}^b_{i,j} - \avg{\vt y}_i) \tr{({\avg{\vt y}}^b_{i,j} - \avg{\vt y}_i)}}{(n_i - b_i)(n_i - b_i + 1)}.
\end{equation}

An issue that remains is to determine a proper block size for each sample $S_i$. Asymptotic consistency requires that $b_i$ increases with $n_i$, and it is a common practice to make $b_i = \lfloor n_i^\nu \rfloor$ for some $0 < \nu < 1$, where $\lfloor \cdot \rfloor$ is the floor operator. Following a recommendation by \citeauthor{Flegal_2010} \cite{Flegal_2010}, here we employ $\nu = 1/2$ and, therefore, $b_i = \lfloor \sqrt{n_i} \rfloor$.

Finally, once $\hat{\mt \Sigma}_{\avg{\vt y}_i}$ has been estimated, the mean square error $\delta^2 \avg A_i$ can be taken as the main-diagonal entry corresponding to $\avg A_i$. In the case of $h(\avg{\vt y}_i)$, the mean square error $\delta^2 h$ might be computed via the delta method \cite{Greene_2012}, from which
\begin{equation}
\label{eq:delta method}
\delta^2 h = \tr{(\nabla_{\avg{\vt y}_i} h)} \hat{\mt \Sigma}_{\avg{\vt y}_i}(\nabla_{\avg{\vt y}_i} h),
\end{equation}
where $\nabla$ is the gradient operator. This expression quantifies the propagation of uncertainties from $\avg{\vt y}_i$ to $h(\avg{\vt y}_i)$.

\subsection{Free-Energy Difference Computation and Property Reweighting}
\label{sec:fep and reweighting}

We are often interested in the difference between the reduced free energies of two states, defined as $\Delta f_{ij} = f_j - f_i = - \ln (Z_j/Z_i)$. Because it is usually unfeasible to compute absolute free energies, we find it appropriate to define $f_i$ as a relative value with respect to a state~$0$, chosen as a common reference to all other states, so that $f_i = -\ln (Z_i/Z_0)$. We can thus rewrite Eq.~\eqref{eq:state_prob_density} as
\begin{equation}
\label{eq:state_prob_density_Z0}
\rho_i(x) = \frac{1}{Z_0} e^{-u_i(x)+ f_i}.
\end{equation}

The value of $f_i$ can only be determined up to an additive constant and this fact will be assumed henceforth. One can estimate $f_i$ by relating it to one or more ensemble averages. We remark, however, that importance sampling usually overlooks the tails of $\rho_i(x)$, so that only a limited region $C_i$ of the configurational space is sampled with reasonable accuracy. Following \citeauthor{Jarzynski_2006} \cite{Jarzynski_2006}, we refer to $C_i$ as the set of \textit{typical} configurations of state $i$. It does not necessarily coincide with the set $D_i$ of configurations that substantially contribute to the integral in Eq.~\eqref{eq:ensemble average}, referred to as the \textit{dominant} configurations. In fact, the estimator in Eq.~\eqref{eq:average estimator} will be accurate only if $D_i$ is a significant subset of $C_i$. This issue is particularly important in the calculation of free-energy differences and their uncertainties.

By using methods such as Free Energy Perturbation (FEP) \cite{Zwanzig_1954}, Simple Overlap Sampling \cite{Lee_1980, Lu_2003}, and Bennett Acceptance Ratio (BAR) \cite{Bennett_1976}, free-energy differences can be computed for those pairs of states whose sets of typical configurations overlap most considerably, and then pieced together to provide differences between poorly- or non-overlapping states. This strategy is known as \textit{staging} \cite{Kofke_1998}. Among the cited methods, FEP is the least recommended one. Given two states $i$ and $j$, it consists in evaluating either $\Delta f_{ij} = -\ln \langle e^{u_i - u_j} \rangle_i$ or $\Delta f_{ij} = \ln \langle e^{u_j - u_i} \rangle_j$ \cite{Zwanzig_1954}. It happens, however, that these averages entail dominant sets $D_i$ and $D_j$ that practically coincide with the typical sets $C_j$ and $C_i$, respectively, which is opposite to the situation that would maximize accuracy \cite{Jarzynski_2006}. On the other hand, BAR is the most accurate method because it manages, in an iterative way, to relate $\Delta f_{ij}$ to ensemble averages computed at states $i$ and $j$ whose dominant sets $D_i$ and $D_j$ fulfill the intersection between $C_i$ and $C_j$.

Another way of tackling overlap deficiency consists in defining a reference state whose set $C_0$ is a contiguous region that encompasses all sets $C_i$ associated to a number of states of interest. Then, instead of sampling configurations at these states, we do it at state $0$, with probability density $\rho_0(x) = \frac{1}{Z_0} e^{-u_0(x)}$, and compute every $f_i$ by means of the perturbation formula
\begin{equation}
\label{eq:nbs sampling free energy}
f_u = -\ln \langle e^{u_0-u} \rangle_0,
\end{equation}
where $f_u$ is the relative free energy of a state whose probability distribution is defined by a reduced potential function $u(x)$.

As explained above, the dominant set $D_0$ will coincide with $C_i$, thus being a subset of $C_0$ as desired. \citeauthor{Torrie_1977} \cite{Torrie_1977} pioneered this idea, known as \textit{non-Boltzmann sampling} due to the usually non-physical character of the sampled state. One is also able to compute, directly from the configurations sampled at state $0$, ensemble averages of any property $A$ at the states of interest. This procedure, known as \textit{reweighting}, is based on the identity \cite{Torrie_1977}
\begin{equation}
\label{eq:nbs sampling reweighting}
\langle A \rangle_u = \frac{\langle A e^{u_0 - u} \rangle_0}{\langle e^{u_0 - u} \rangle_0}.
\end{equation}

The multicanonical approach \cite{Berg_1992, Lee_1993, Abreu_2006} and the enveloping distribution method \cite{Christ_2007, Christ_2008, Christ_2009} are examples of successful non-Boltzmann sampling techniques.

\subsection{Expanded Ensemble as a Mixture Model}
\label{sec:expanded ensemble}

A method of particular relevance is the one known as \textit{expanded ensemble} \cite{Lyubartsev_1992}. It is a non-Boltzmann sampling scheme, but closely related to staging as it consists in combining a number $m$ of states whose distributions overlap. However, instead of sampling each state independently, we perform importance sampling by following a joint probability of configurations and states given by \cite{Nymeyer_2010}
\begin{equation}
\label{eq:expanded ensemble joint}
p_0(x, i) = \frac{1}{Z_0} e^{-u_i(x) + \eta_i},
\end{equation}
where $\vt \eta$ is a vector of arbitrary weighting factors. This procedure results in a sample $S = \{x_k,i_k\}_{k=1}^n$, where each configuration bears a state-indicating label. Different ways of performing such sampling can be found in the literature \cite{Lyubartsev_1992, Nymeyer_2010, Christ_2007, Christ_2008, Christ_2009, Katzgraber_2006, Trebst_2006, Escobedo_2007, Escobedo_2008, Martinez_2008, Chodera_2011_2, Ding_2017}. From Eq.~\eqref{eq:expanded ensemble joint}, one can derive the marginal probability of each state $i$, defined as $\pi_i = \int p_0(x,i)dx$, as well as the marginal probability density at state~$0$, defined as $\rho_0(x) = \sum_{i=1}^m p_0(x,i)$. One can also obtain the conditional probabilities $p(i|x)$ and $p(x|i)$ via Bayes' theorem, $p_0(x,i) = p(x|i) \pi_i = p(i|x) \rho_0(x)$. Hence,
\begin{subequations}
	\label{eq:expanded ensemble probabilities}
	\begin{gather}
	\pi_i = \frac{Z_i}{Z_0} e^{\eta_i} = e^{\eta_i - f_i}, \label{eq:expanded ensemble prior} \\
	\rho_0(x) = \frac{1}{Z_0} \sum_{j=1}^m e^{-u_j(x) + \eta_j}, \label{eq:expanded ensemble evidence} \\
	p(i|x) = \frac{e^{-u_i(x) + \eta_i}}{\sum_{j=1}^m e^{-u_j(x) + \eta_j}}, \; \text{and} \label{eq:expanded ensemble posterior} \\
	p(x|i) = \frac{1}{Z_0} e^{-u_i(x) + f_i}. \label{eq:expanded ensemble likelihood}
	\end{gather}
\end{subequations}

The fact that $p(x|i) = \rho_i(x)$ demonstrates that each individual state is sampled correctly. Once the sample $S$ has been generated, a simple estimator for the relative free energy of each state stems from Eq.~\eqref{eq:expanded ensemble prior} by making $\hat \pi_i = \frac{n_i}{n}$, where $n_i$ is the number of appearances of state $i$ in $S$. The result is
\begin{equation}
\label{eq:expanded ensemble histogram estimator}
\hat f_i = \eta_i - \ln \frac{n_i}{n}.
\end{equation}

However, we can derive a much better (i.e. Rao-Blackwellized \cite{Carlson_2016, Ding_2017}) estimator if we realize that $\pi_i = \int \rho_0(x) p(i|x) dx$, meaning that $\pi_i$ is the ensemble average of the marginal probability $p(i|x)$ at state $0$. The new estimator, which arises if we simply use Eq.~\eqref{eq:average estimator} to estimate such an average, is
\begin{equation}
\label{eq:expanded ensemble FEP estimator}
\hat f_i = -\ln \left[ \frac{1}{n}\sum_{k=1}^n \frac{e^{-u_i(x_k)}}{\sum_{j=1}^m e^{-u_j(x_k) + \eta_j}} \right].
\end{equation}

Note that the state-indicating labels in $S$ are neglected in both estimators, due to an essential feature of importance sampling: every sampled configuration contributes equally to an ensemble average. In practice, the set of weighting factors is often chosen so that all states are visited with equal frequency. This requires that $\eta_i = f_i - \ln m$, which can only be achieved iteratively. Other strategies have also been proposed\cite{Katzgraber_2006, Trebst_2006, Escobedo_2007, Escobedo_2008, Martinez_2008} for the selection of suitable weighting factors or reduced potentials in order to improve performance and avoid ergodicity issues.

Finally, we can combine the results in Eq.~\eqref{eq:expanded ensemble probabilities} in order to show that an expanded ensemble distribution is actually a \textit{mixture model} \cite{Lindsay_1995, Marin_2005} composed of the distributions corresponding to the $m$ states, that is,
\begin{equation}
\label{eq:mixture ensemble}
\rho_0(x) = \sum_{i=1}^m \pi_i \rho_i(x).
\end{equation}

In this context, the vector $\vt \pi$ of state probabilities can be regarded as the mixture composition. Moreover, since $\rho_0 = \frac{1}{Z_0}e^{-u_0}$, we can resort to Eq.~\eqref{eq:state_prob_density_Z0} for expressing the reduced potential of the mixture state as
\begin{equation}
\label{eq:mixture potential}
u_0(x) = -\ln \sum_{i=1}^m \pi_i e^{-u_i(x) + f_i}.
\end{equation}

With $\vt f$ replaced by its estimator $\hat{\vt f}$, this potential can be employed with Eqs.~\eqref{eq:nbs sampling free energy} and \eqref{eq:nbs sampling reweighting} to estimate relative free energies and other ensemble averages at any state whose set $C$ is a subset of $C_0$, whether such state is part of the expanded ensemble or not. As did \citeauthor{Geyer_1994} \cite{Geyer_1994} and \citeauthor{Shirts_2017} \cite{Shirts_2017}, we refer to this procedure as ``reweighting from the mixture''. A remarkable feature of this approach is that Eqs.~\eqref{eq:nbs sampling free energy}, \eqref{eq:nbs sampling reweighting}, and \eqref{eq:expanded ensemble FEP estimator} could all be rewritten so that only differences in reduced potentials were present. Thus, if we are able to write $u(x) = \nu(x) + \phi(x)$, where $\nu(x)$ is identical for all states of interest, then only the distinguishing term $\phi(x)$ is relevant in the calculations.

\subsection{The Multistate Bennett Acceptance Ratio Method}

An interesting feature of an expanded ensemble is that the walk throughout states favors ergodicity, as compared to sampling each state individually. Nonetheless, the process of determining convenient states and weights is often tedious and time-consuming. It is thus more likely that we face the situation described in Sec.~\ref{sec:definitions}, when samples individually drawn at different states are available. An advantage of this scenario is that inclusion of new states, if necessary, is straightforward. In this case, the pooled sample can be expressed as $\mathcal S = \big\{\{x_{i,k}\}_{k=1}^{n_i} \big\}_{i=1}^m$. \citeauthor{Shirts_2008} \cite{Shirts_2008} described a method for computing relative free energies and their uncertainties by using all configurations in $\mathcal S$, provided that each sample is composed of i.i.d. configurations. The authors named the method as Multistate Bennet Acceptance Ratio (MBAR) because it is equivalent to BAR when only two states are involved. Let us derive their estimator from a simple argument before embarking in a detailed analysis. MBAR consists in assigning the same importance to every configuration, regardless of which state it comes from. This makes $\mathcal S$ equivalent to the outcome of an expanded ensemble simulation, so that we can overlook its subset structure and write $\mathcal S = \{x_k\}_{k=1}^n$. The issue of not knowing the weighting factors of this expanded ensemble is easily tackled by combining the two estimators of Eqs.~\eqref{eq:expanded ensemble histogram estimator} and \eqref{eq:expanded ensemble FEP estimator}. Elimination of $\vt \eta$ yields
\begin{equation}
\label{eq:mbar free energy estimator}
\hat f_i = -\ln \sum_{k=1}^n \frac{e^{-u_i(x_k)}}{\sum_{j=1}^m n_j e^{-u_j(x_k) + \hat f_j}},
\end{equation}
which is equivalent to Eq.~11 of Ref.~\citenum{Shirts_2008}. This is not an isolated estimator, but a self-consistent system of equations whose solution provides the whole vector $\hat {\vt f}$. In fact, because the solution can only be determined up to an additive constant, one can fix $\hat f_1$ and solve the remaining equations in order to obtain $\hat{\vt f}_{2:m}$. After that, we can use $\hat{\vt f}$ to perform FEP and reweighting calculations, exactly as described in the preceding section.

As the $m$ samples in $\mathcal S$ are generated independently, giving all configurations the same importance seems unsuitable if these samples exhibit correlation and distinct mixing times. This is one reason why MBAR is meant to deal with iid samples and requires subsampling whenever some sample is originally correlated \cite{Shirts_2008}. Another important reason is concerned with the determination of uncertainties. The MBAR estimator had previously appeared in the literature under other denominations such as Biased Sampling \cite{Vardi_1985, Gill_1988}, Reverse Logistic Regression \cite{Geyer_1994}, and Extended Bridge Sampling \cite{Meng_1996, Kong_2003, Tan_2004}. It has been derived in several ways from the principle of maximum likelihood, which can lead to standardized proofs of CLT's, thus providing reliable estimates of asymptotic covariance matrices \cite{Pawitan_2001, Greene_2012}. These proofs, however, are usually based on the premise that i.i.d. samples are available. More details on subsampling and uncertainty estimation are given in Appendix~\ref{sec:subsampling and uncertainty in MBAR}.

Finally, we remark that our interpretation of MBAR as an expanded ensemble with unknown weights is closely related to recent contributions from \citeauthor{Shirts_2017} \cite{Shirts_2017}, who presented the method in a revised form (based on the work of \citeauthor{Geyer_1994} \cite{Geyer_1994}) and from \citeauthor{Ding_2017} \cite{Ding_2017}, who proposed an alternative derivation of the method's main equation.

\section{Taking Full Advantage of Multiple Time-Correlated Samples}

The requirement of subsampling often causes MBAR to waste a considerable amount of data. Even though the discarded configurations carry less information than do the ones that remain, they could possibly have a positive impact on the quality of a statistical analysis. In what follows, we develop an unwasteful extension of MBAR, which allows one to use all available data even if individual samples exhibit distinct mixing times.

\subsection{Relative Free Energies of Sampled States}

The starting point of our proposal is an exact expression obtained from Eqs.~\eqref{eq:nbs sampling free energy} and \eqref{eq:mixture potential}, which is
\begin{equation}
\label{eq:free energy exact}
f_i = -\ln \left\langle \frac{e^{-u_i}}{\sum_{j=1}^m \pi_j e^{-u_j + f_j}} \right\rangle_0.
\end{equation}

In MBAR, the average above is estimated by applying Eq.~\eqref{eq:average estimator} to the overall sample $\mathcal S$. Such decision is the source of all the restraints discussed in the preceding section. This is so because all configurations in $\mathcal S$ must be identically distributed according to $\rho_0(x)$ so that both $\vt \pi$ and $\vt f$ can be properly determined. Nevertheless, Eqs.~\eqref{eq:ensemble average} and \eqref{eq:mixture ensemble} show that, for a mixture model, $\langle A \rangle_0 = \sum_{i=1}^m \pi_i \langle A \rangle_i$. Therefore, one can estimate an average at the mixture state from averages at the individual ones, that is,
\begin{equation}
\label{eq:mixture average estimator}
{\avg A}_0 = \sum_{i=1}^m \pi_i \avg{A}_i = \sum_{i=1}^m \frac{\pi_i}{n_i} \sum_{k=1}^m A(x_{i,k}).
\end{equation}

We will refer to this estimator as a \textit{mixture of independently collected samples} (MICS). It requires no assumption on the distribution of the pooled sample. Moreover, it is valid for any mixture composition $\vt \pi$, which can then be regarded as a set of prior probabilities (i.e. specified beforehand). This makes it easy to combine OBM-based covariance matrices like $\hat{\mt \Sigma}_{\avg{\vt y}_i}$, obtained via Eq.~\eqref{eq:obm asymptotic covariance} at each state $i$, into a matrix of covariances between averages computed at the mixture state. Because $\avg{\vt y}_0 = \sum_{i=1}^m \pi_i \avg{\vt y}_i$ and each average $\avg{\vt y}_i$ is independent (i.e. covariance is null for all $j \neq i$), the fact that every $\pi_i$ is known exactly results in
\begin{equation}
\label{eq:mixture variance estimator}
\hat{\mt \Sigma}_{\avg{\vt y}_0} = \sum_{i=1}^m \pi_i^2 \hat{\mt \Sigma}_{\avg{\vt y}_i}.
\end{equation}

Of course, the choice of $\vt \pi$ is expected to affect the asymptotic behavior of the MICS estimator, and this will be a topic of discussion shortly. Finally, the equation we propose for estimating the relative free energies of the sampled states, derived from applying the MICS estimator to Eq.~\eqref{eq:mixture average estimator}, is
\begin{equation}
\label{eq:mblock free energy estimator}
{\hat f}_i = -\ln \sum_{j=1}^m \frac{\pi_j}{n_j} \sum_{k=1}^{n_j} \frac{e^{-u_i(x_{j,k})}}{\sum_{s=1}^m \pi_s e^{-u_s(x_{j,k}) + {\hat f}_s}}.
\end{equation}

As with MBAR, we again have a system of equations that must be solved self-consistently after setting the value of $\hat f_1$, for instance. In contrast, however, the state at which each configuration was sampled is now a relevant information, unless we specify $\pi_j = {n_j}/{n}$ in order to recover Eq.~\eqref{eq:mbar free energy estimator}. Extending an approach pioneered by \citeauthor{Geyer_1994} \cite{Geyer_1994}, \citeauthor{Doss_2014} \cite{Doss_2014} proposed an equivalent estimator and investigated its asymptotic behavior when covariances at individual states are computed via regenerative simulations. In a subsequent study, \citeauthor{Roy_2018} \cite{Roy_2018} extended their findings to the case of covariances based on non-overlapping batch-means. They were also able to soften the conditions required for proving the consistency of the estimator. Here we assume the observance of these conditions and adopt the OBM strategy described in Sec.~\ref{sec:definitions}. 

As \citeauthor{Roy_2018} \cite{Roy_2018} suggested, an adequate specification for the marginal probabilities $\vt \pi$ should include some definition of effective sample size (ESS). A straightforward choice is
\begin{equation}
\label{eq:mblock prior}
\pi_i = \frac{n^\text{eff}_i}{\sum_{j=1}^m n^\text{eff}_j},
\end{equation}
where $n^\text{eff}_i$ is the effective size of the sample obtained at state $i$. It is possible to employ a multivariate analysis to estimate the ESS, such as in the way \citeauthor{Vats_2015} \cite{Vats_2015, Vats_2018} have recently proposed. Traditionally, however, a single variable is chosen for such task. Here, unless stated otherwise, for a sample collected at state $i$ we choose the reduced energy $u_i$ (that is, the one that defines that same state) and compute the ESS as
%\begin{equation*}
%n^\text{eff}_i = n_i \left[\frac{\det(\hat{\mt \Sigma}_{\avg{\vt u}_i})}{\det(\hat{\mt \Sigma}^\ast_{\avg{\vt u}_i})}\right]^{1/m}
%\end{equation*}
\begin{equation*}
n^\text{eff}_i = n_i \frac{\hat{\Sigma}_{\avg{u_i}_i}}{\hat{\Sigma}^\ast_{\avg{u_i}_i}},
\end{equation*}
where the variances $\hat{\Sigma}_{\avg{u_i}_i}^\ast$ and $\hat{\Sigma}_{\avg{u_i}_i}$ are computed via Eqs.~\eqref{eq:asymptotic covariance IID} and \eqref{eq:obm asymptotic covariance}, respectively. As it will be shown in Sec.~\ref{sec:solvation free energy}, the MICS estimator is not much sensible to how precise the ESS is computed.

\subsection{Maximum Likelihood Approach and Numerical Computation}
\label{sec:maximum likelihood}

Following \citeauthor{Doss_2014} \cite{Doss_2014}, we can derive the non-linear system in Eq.~\eqref{eq:mblock free energy estimator} by a maximum likelihood approach, based on the log-quasi-likelihood function \cite{Doss_2014, Tan_2015, Roy_2018}
\begin{equation}
\label{eq:mblock log-quasi-likelihood}
\ln \mathcal L = \sum_{i=1}^m \frac{\pi_i}{n_i} \sum_{k=1}^{n_i} \ln p_i(x_{i,k}),
\end{equation}
where $p_i(x)$ is the conditional probability $p(i|x)$, obtained by combining Eqs.~\eqref{eq:expanded ensemble prior} and \eqref{eq:expanded ensemble posterior}, that is,
\begin{equation}
\label{eq:mixture posterior probability}
p_i(x) = \frac{\pi_i e^{-u_i(x) + f_i}}{\sum_{j=1}^m \pi_j e^{-u_j(x) + f_j}}.
\end{equation}

The first- and second-order derivatives of $\ln p_k(x)$ with respect to $f_i$ and $f_j$ are, respectively,
\begin{equation*}
\frac{\partial \ln p_k}{\partial f_i} = \delta_{i,k} - p_i
\end{equation*}
and
\begin{equation*}
\frac{\partial^2\ln p_k}{\partial f_i \partial f_j} = -p_i(\delta_{i,j} - p_j),
\end{equation*}
where $\delta_{i,j}$ is the Kronecker delta. It is now convenient to introduce a vector/matrix notation. First, we define a vector-valued configurational property $\vt p(x)$, whose each entry $i$ is $p_i(x)$, given by Eq.~\eqref{eq:mixture posterior probability}. Next, a matrix-valued property $\mt B(x)$ is defined, where $\mt B = \text{diag}(\vt p) - {\vt p}\tr{\vt p}$. We remark that $\mt B$ is both symmetric and singular, as it satisfies $\tr{\mt B} = \mt B$ and $\mt B\vt 1 = \vt 0$, where $\vt 1$ is a vector of ones and $\vt 0$ is a vector of zeros. Note that the MICS estimator naturally extends to multivalued properties, so that we can use Eq.~\eqref{eq:mixture average estimator} to compute averages like $\avg{\vt p}_0$ and $\avg{\mt B}_0$. Then, the reciprocal gradient and reciprocal Hessian of the log-quasi-likelihood function with respect to the relative free-energy vector $\vt f$ are, respectively,
\begin{equation}
\label{eq:mblock score vector}
\vt g = -\nabla_{\vt f} \ln \mathcal L = \avg{\vt p}_0 - \vt \pi
\end{equation}
and
\begin{equation}
\label{eq:mblock fisher information matrix}
\mt H = -\nabla^2_{\vt f} \ln \mathcal L = \avg{\mt B}_0.
\end{equation}

For a given pooled sample $\mathcal S$ and predefined mixture composition $\vt \pi$, both $\vt g$ and $\mt H$ are exclusive functions of $\vt f$. Thus, the maximum likelihood estimator is the solution of $\vt g(\hat{\vt f}) = \vt 0$, which occurs when $\avg{\vt p}_0 = \vt \pi$, that is, when the computed expectation of $\vt p(x)$ at the mixture state equals the specified mixture composition. This is just a reformulation of Eq.~\eqref{eq:mblock free energy estimator}, provided that $\avg{\vt p}_0$ is defined as in Eq.~\eqref{eq:mixture average estimator}. Another interesting interpretation is possible by defining a stochastic matrix $\mt M = [\begin{array}{ccc} \avg{\vt p}_1 & \cdots & \avg{\vt p}_m \end{array}]$, where each $\avg{\vt p}_i$ is computed via Eq.~\eqref{eq:average estimator}. It turns out that $\mt M = \tr{\mt O}$, where $\mt O$ is the \textit{overlap matrix} which \citeauthor{Klimovich_2015} \cite{Klimovich_2015} defined as a tool for quantifying phase-space overlap in multistate methods. Such definition allows us to write $\mt M \vt \pi = \vt \pi$ as yet another form for Eq.~\eqref{eq:mixture average estimator}, meaning that the estimator $\hat{\vt f}$ is the one that causes the overlap matrix to have a stationary distribution that coincides with the mixture composition.

Application of Eq.~\eqref{eq:mixture average estimator} to compute both $\vt g$ and $\mt H$ requires the vector of relative free energies. One can find an initial guess by doing direct calculations (e.g. via overlap sampling\cite{Lee_1980, Lu_2003}) for pairs of adjacent states with respect to the averages of their own reduced potentials (i.e. $\langle u_i \rangle_i$). The solution of Eq.~\eqref{eq:mblock score vector} can be found via Newton-Raphson iterations in the form
\begin{equation*}
\label{eq:mics Newton-Raphson}
\hat{\vt f} \leftarrow \hat{\vt f} - \avg{\mt B}_0^+ (\avg{\vt p}_0 - {\vt \pi}),
\end{equation*}
where $\avg{\mt B}_0^+ $ is the Moore-Penrose pseudoinverse of $\avg{\mt B}_0$, needed because $\avg{\mt B}_0$ is rank-deficient. To prevent drift in the solution, every update step should include doing $\hat{\vt f} \leftarrow \hat{\vt f} - \hat{f}_1 {\vt 1}$, thus enforcing that $\hat f_1 = 0$. The log-quasi-likelihood function in Eq.~\eqref{eq:mblock log-quasi-likelihood} is concave \cite{Doss_2014}, which makes the solution above unique. In addition, $\hat{\vt f}$ is a consistent estimator \cite{Doss_2014}, meaning that it converges to the true vector $\vt f$ when $n \rightarrow \infty$ with fixed ratios $n_i/n$ for all $i$. Employing the pseudoinverse $\avg{\mt B}_0^+ $ (instead of solving a smaller system in which the first row and column of $\avg{\mt B}_0$ are omitted \cite{Shirts_2008}) allows additional rank-reduction factors such as using two or more samples drawn at the same state. Besides, $\avg{\mt B}_0^+ $ plays a central role in the estimation of uncertainties, as it will become clear in the forthcoming subsections.

\subsection{Uncertainties of Free-Energy Differences}
\label{sec:uncertainties of free-energy differences}

As one can deduce from the delta method, Eq.~\eqref{eq:delta method}, the mean square error of a free energy difference $\Delta \hat f_{i,j} = \hat f_j - \hat f_i$ can be calculated by
\begin{equation}
\delta^2 \Delta f_{i,j} = \hat \Theta_{i,i} + \hat \Theta_{j,j} - 2\hat \Theta_{i,j},
\end{equation}
where $\hat{\mt \Theta} = \hat{\mt \Sigma}_{\hat{\vt f}}$ is the matrix of estimated asymptotic covariances among the elements of $\hat{\vt f}$. We propose a simple reasoning that leads to the correct equation for computing $\hat{\mt \Theta}$ from the pooled sample $\mathcal S$, according to CLT proofs found in the literature \cite{Geyer_1994, Buta_2010, Buta_2011, Doss_2014}. For this, we first extend Eq.~\eqref{eq:delta method} to account for the propagation of uncertainties from a vector $\vt y$ to a vector-valued function $\vt h({\vt y})$. In this case, the result is no longer a single value, but covariance matrix given by
\begin{equation}
\label{eq:vectorial delta method}
\hat{\mt \Sigma}_{\vt h} = \tr{(\nabla_{\vt y} \tr{\vt h})} \hat{\mt \Sigma}_{\vt y}(\nabla_{\vt y} \tr{\vt h}),
\end{equation}
whose main diagonal contains the vector of mean square errors of $\vt h$. It is evident from Eqs.~\eqref{eq:mblock score vector} and \eqref{eq:mblock fisher information matrix} that $\nabla_{\vt f}\tr{\avg{\vt p}}_0 = \nabla_{\vt f}\tr{\vt g} = \avg{\mt B}_0$. Thus, from the delta method and the symmetry of $\avg{\mt B}_0$, it follows that $\hat{\mt \Sigma}_{\avg{\vt p}_0} = \avg{\mt B}_0 \hat{\mt \Theta} \avg{\mt B}_0$. This is an equation for $\hat{\mt \Theta}$ because $\hat{\mt \Sigma}_{\avg{\vt p}_0}$ can be easily estimated via Eqs.~\eqref{eq:obm asymptotic covariance} and \eqref{eq:mixture variance estimator}. Due to the singularity of $\avg{\mt B}_0$ (as discussed in Sec.~\ref{sec:maximum likelihood}), a feasible solution for this equation is
\begin{equation}
\label{eq:mblock covariance matrix}
\hat{\mt \Theta} = \avg{\mt B}_0^+  \hat{\mt \Sigma}_{\avg{\vt p}_0} \avg{\mt B}_0^+.
\end{equation}

\subsection{Reweighting from the Mixture}

Suppose that we are interested in computing the ensemble average of a vector-valued property $\vt y(x)$ at a state whose reduced potential is an arbitrary function $u(x)$. After determining $\hat{\vt f}$, one can use Eq.~\eqref{eq:nbs sampling reweighting} and the MICS equation in Eq.~\eqref{eq:mixture average estimator} to estimate $\langle \vt y \rangle_u$ by
\begin{equation}
\label{eq:reweighting from mixture}
\avg{\vt y}_u = \frac{\avg{\vt z}_0}{\avg{w}_0},
\end{equation}
where $w(x) = e^{u_0(x) - u(x)}$ and ${\vt z}(x) = w(x) {\vt y}(x)$, with the reduced potential $u_0$ coming from Eq.~\eqref{eq:mixture potential}.

The uncertainty of $\avg{\vt y}_u$ stems from several sources. It clearly depends on the asymptotic variance $\hat \Sigma_{\avg w_0}$ and covariance matrix $\hat{\mt \Sigma}_{\avg{\vt z}_0}$, as well as on the cross-covariance between $\avg{\vt z}_0$ and $\avg w_0$. In addition, $u_0$ can only be computed using an estimated vector $\hat{\vt f}$, whose uncertainty propagates to $\avg{\vt y}_u$ as well. In the context of Bayesian analysis, \citeauthor{Buta_2011} \cite{Buta_2011} suggested that $\hat{\vt f}$ and $\avg{\vt y}_u$ should be estimated using distinct collections of samples. This approach, which has been followed in subsequent work \cite{Tan_2015, Roy_2018}, would save us from having to consider the cross-covariances amongst $\hat{\vt f}$, $\avg{\vt z}_0$, and $\avg w_0$. However, this is not convenient in the context of molecular simulation due to the massive computational effort needed to generate the samples. Here we propose a simple way of including all covariances, based again on the one-to-one relation between $\hat{\vt f}$ and $\avg{\vt p}_0$. We start by piling together $\vt p$, $\vt z$, and $w$ (in this order) into a unique vector-valued property $\vt s$. This allows us to estimate $\hat{\mt \Sigma}_{\avg{\vt s}_0}$ directly from the samples by using the OBM method and the MICS estimator in Eq.~\eqref{eq:mixture variance estimator}. After that, we can use the delta method to obtain the asymptotic covariance matrix $\hat{\mt \Theta}_{\avg{\vt y}_u}$ as
\begin{subequations}
\label{eq:mics reweighting}
\begin{equation}
\label{eq:mics reweighting delta method}
\hat{\mt \Sigma}_{\avg{\vt y}_u} = \tr{(\nabla_{\avg{\vt s}_0} {\tr{\avg{\vt y}}_u})} \hat{\mt \Sigma}_{\avg{\vt s}_0} (\nabla_{\avg{\vt s}_0} {\tr{\avg{\vt y}}_u}),
\end{equation}
where
\begin{equation}
\label{eq:mics reweighting gradient}
\nabla_{\avg{\vt s}_0} \tr{\avg{\vt y}}_u = \left[\begin{array}{c}
\avg{\mt B}_0^+ [\avg{\vt p}_u \tr{\avg{\vt y}}_u - \avg{(\vt p \tr{\vt y})}_u] \\
\avg{w}_0^{-1} {\mt I} \\
-\avg{w}_0^{-1} \tr{\avg{\vt y}}_u
\end{array}\right].
\end{equation}
\end{subequations}

The two lower blocks of the matrix above are, respectively, $\nabla_{\avg{\vt z}_0} \tr{\avg{\vt y}}_u$ and $\nabla_{\avg{w}_0} \tr{\avg{\vt y}}_u$, obtained by direct transposition and differentiation of Eq.~\eqref{eq:reweighting from mixture}. The upper block is $\nabla_{\avg{\vt p}_0} \tr{\avg{\vt y}}_u$, which is linked to $\nabla_{\vt f} \tr{\avg{\vt y}}_u$ via the chain rule. The latter derivative is given by
\begin{equation*}
\begin{split}
\nabla_{\vt f} \tr{\avg{\vt y}}_u &= (\nabla_{\vt f} \tr{\avg{\vt z}}_0) (\nabla_{\avg{\vt z}_0} \tr{\avg{\vt y}}_u) + (\nabla_{\vt f} {\avg{w}_0}) (\nabla_{\avg{w}_0} \tr{\avg{\vt y}}_u) = \\
&= \frac{{\avg{(\nabla_{\vt f} \tr{\vt z})}_0}}{\avg{w}_0} - \frac{\avg{(\nabla_{\vt f} w)}_0}{\avg{w}_0} \tr{\avg{\vt y}}_u.
\end{split}
\end{equation*}

Subsequently, it follows from the definition of $w$ that $\nabla_{\vt f} w = w (\nabla_{\vt f} u_0)$ and from the chain rule that $\nabla_{\vt f} \tr{\vt z} = (\nabla_{\vt f} w) (\nabla_w \tr{\vt z}) = w (\nabla_{\vt f} u_0) \tr{\vt y}$. By differentiating Eq.~\eqref{eq:mixture potential} with respect to $f_i$ and then contrasting the result with Eq.~\eqref{eq:mixture posterior probability}, it turns out that $\nabla_{\vt f}u_0 = -{\vt p}$. Therefore,
\begin{equation*}
\nabla_{\vt f} \tr{\avg{\vt y}}_u = \frac{\avg{(w {\vt p})}_0}{\avg{w}_0} \tr{\avg{\vt y}}_u - \frac{\avg{(w{\vt p}\tr{\vt y})}_0}{\avg{w}_0}.
\end{equation*}

It is clear from Eq.~\eqref{eq:reweighting from mixture} that both ratios above are ensemble averages reweighted to the state defined by $u(x)$. Then, a procedure like the one described in Sec.~\ref{sec:uncertainties of free-energy differences} yields $\nabla_{\avg{\vt p}_0} \tr{\avg{\vt y}}_u = \avg{\mt B}_0^+ (\nabla_{\vt f} \tr{\avg{\vt y}}_u)$, which finally proves the identity in Eq.~\eqref{eq:mics reweighting}.

As already mentioned, the mean square error of $\avg{\vt y}_u$ is the main diagonal of $\hat{\mt \Sigma}_{\avg{\vt y}_u}$. In addition, uncertainties in combinations of entries of $\avg{\vt y}_u$ will involve non-diagonal entries of the matrix.

Discuss about the link of the equation above and Remark 2 of Theorem 1 in Buta.\cite{Buta_2010}

Discuss on the the hidden assumption of MBAR, which neglects the propagation of uncertainty from $\hat{\vt f}$ to $\avg{\vt y}_u$.

\subsection{Free-Energy Perturbation}

\begin{equation*}
\Delta f_{u,r} = -\ln \avg{w}_0 - f_r
\end{equation*}

\begin{equation*}
\nabla_{\avg{\vt p}_0}{f_u} = \avg{\mt B}_0^+  \nabla_{\vt f}{f_u} = -\avg{\mt B}_0^+  \left({\frac{\nabla_{\vt f} \avg{w}_0}{\avg{w}_0}} + {\vt e}_r \right)
\end{equation*}

\begin{equation*}
\nabla_{\avg{\vt p}_0}{f_u} = \avg{\mt B}_0^+  (\avg{\vt p}_u - {\vt e}_r).
\end{equation*}


\begin{equation*}
\nabla_{\avg{w}_0}{f_u} = \avg{w}_0^{-1}
\end{equation*}


\subsection{Derivatives?}

Given $u(x,\lambda)$, calculate $\frac{\partial f_u}{\partial \lambda}$ and $\frac{\partial \langle \vt y \rangle_u}{\partial \lambda}$. Considering that $\nu = \diff{u}{\lambda}$

\begin{equation*}
\diff{\langle \vt y \rangle_u}{\lambda} \approx \avg{\omega}_0^{-1} \left( \diff{\avg{\vt z}_0}{\lambda} - \avg{\vt y}_u \diff{\avg{\omega}_0}{\lambda} \right).
\end{equation*}

\begin{equation*}
\diff{\omega}{\lambda} = -\omega \diff{u}{\lambda} = -\omega(x) \nu(x)
\end{equation*}


\begin{equation*}
\diff{\avg{\vt z}_0}{\lambda} = \avg{\left( \diff{\omega}{\lambda} \vt y \right)} = -\avg{(\omega u^\prime \vt y)}_0
\end{equation*}

Finally,
\begin{equation*}
\diff{\avg{\vt y}_u}{\lambda} = \avg{\nu}_u \avg{\vt y}_u - \avg{\nu \vt y}_u
\end{equation*}


\begin{equation*}
\diff{f_u}{\lambda} = -\avg{\omega}_0^{-1} \diff{\avg{\omega}_0}{\lambda} = \avg{\nu}_u
\end{equation*}

\subsection{Implementation?}


\section{Results}

Although it is usual to employ toy models for evaluating a new method, we choose to do it by employing real-world examples.

In order to validate the proposed method and compare its performance with those of well-known methods such as WHAM and MBAR, in this section we carry out real-world simulations.

\subsection{Hydration Free Energy of Glucose}
\label{sec:solvation free energy}

We performed Molecular Dynamics simulations of a single $\alpha$-D-glucose molecule surrounded by 1500 water molecules in a cubic box with periodic boundary conditions. The force fields GLYCAM-06 \cite{Kirschner_2007} and TIP4P-2005 \cite{Abascal_2005} were employed to model the solute and the solvent, respectively. Lennard-Jones parameters for van der Waals interactions between unlike atoms were obtained from the Lorentz-Berthelot mixing rule. Electrostatic interactions were computed in a pairwise fashion, without resorting to Ewald sums. Within the glucose molecule, neither van der Waals nor electrostatic interactions were computed for pairs of atoms bonded together (1-2) or bonded to a common partner (1-3), while Lennard-Jones and full Coulomb interactions were computed for all other pairs. This is in accordance with the GLYCAM-06 \cite{Kirschner_2007} force field. In the case of TIP4P-2005 \cite{Abascal_2005} water, molecules are modeled as rigid bodies and, therefore, no intramolecular interactions are relevant at all. For the intermolecular electrostatics, the Coulomb's Law expression was multiplied by a damping factor $\text{erfc}(-\alpha r)$, with $\alpha = 0.2~\text{\AA}^{-1}$, followed by a shifted-force truncation \cite{Allen_1987}. The suitability of using this approach as an alternative to Ewald sums for isotropic systems has been discussed in the literature \cite{Fennell_2006}. In addition, electrostatic interactions between water and glucose atoms was multiplied by a coupling parameter $\lambda_\text{C}$, so that they could be turned off completely ($\lambda_\text{C} = 0$) or partially ($0 < \lambda_\text{C} < 1$). In turn, van der Waals interactions between water and glucose atoms were calculated by means of the softcore potential of \citeauthor{Beutler_1994} \cite{Beutler_1994}, given by $\nu(r) = 4\epsilon\lambda_\text{vdW}(s^{-2} - s^{-1})$, where $s = (r/\sigma)^6 + (1-\lambda_\text{vdW})/2$. In this way, by tunning $\lambda_\text{vdW}$ in the inteval $0 \le \lambda_\text{vdW} \le 1$ we could smoothly transform them from null to Lennard-Jones interactions. Cutoff distances of $9~\text{\AA}$ and $12~\text{\AA}$ were considered for water-water and water-glucose atom pairs, respectively, with standard tail corrections employed for the van der Waals contribution.

The molecular dynamics simulations were executed using the software package LAMMPS \cite{Plimpton_1995} with customized pair styles and employing the NO\_SQUISH algorithm \cite{Dullweber_1997, Miller_2002, Silveira_2017} to deal with rigid body motion. Besides, we invoked the SHAKE algorithm \cite{Ryckaert_1977} for fixing the lengths of all bonds in the glucose molecule which involve hydrogen atoms. In each run, an isothermal-isobaric ensemble at $300~\text{K}$ and $1~\text{bar}$ was simulated by using two separate thermostat chains to control the temperature for the rigid solvent\cite{Kamberaj_2005} and the flexible solute\cite{Martyna_1994} separately, while a Martyna-Tobias-Klein barostat\cite{Martyna_1994} was responsible for controlling the system pressure. Relaxation time constants of $100~\text{fs}$ and $1000~\text{fs}$ were set for the thermostats and for the barostat, respectively. In order to minimize the effect of discretization errors in the sampled distributions (see Refs.~\citenum{Davidchack_2012} and \citenum{Silveira_2017}), we employed a small time step of $1~\text{fs}$. Each run comprised $1~\text{ns}$ of equilibration time and $5~\text{ns}$ of production time, with configurational properties being collected at every 20 time steps.

Here we break the solvation free energy calculation into two parts. While keeping $\lambda_\text{C} = 0$ (i.e. all water-glucose electrostatic interactions turned off), we execute 8 independent simulations with different values of $\lambda_\text{vdW}$ from 0 to 1. In parallel, while keeping $\lambda_\text{vdW} = 1$ (i.e. all water-glucose van der Waals interactions given by Lennard-Jones terms), we execute 6 simulations with values of $\lambda_\text{C}$ varying from 0 to 1. According to \citeauthor{Shirts_2003_2} \cite{Shirts_2003_2}, if we neglect the change in average volume that results from adding a glucose molecule to the system, the solvation free energy can be computed by ${\Delta g}_\text{solv} = RT (\Delta f_\text{vdW} + \Delta f_\text{C})$, where $\Delta f_\text{vdW}$ and $\Delta f_\text{C}$ are the total free-energy variations involved in the first and second steps described above, respectively.


\begin{table*}
	\caption{Free energy differences. Decorrelation via derivatives $dE/d\lambda_\text{vdW}$ from state 1 to state 8, $dE/d\lambda_\text{C}$ afterwards.}
	\label{table:glucose Coulomb free energies}
	\begin{tabular}{CCCCCCCCC}
		\hline
		\text{state} & \lambda_\text{vdW} & \lambda_\text{C} & n^\text{eff}_\text{\tiny MICS} & \Delta f_\text{\tiny MICS} & \delta \Delta f_\text{\tiny MICS} & n_\text{\tiny MBAR} & \Delta f_\text{\tiny MBAR} & \delta \Delta f_\text{\tiny MBAR} \\
		\hline
		1 & 0.00 & 0.0 & 2960 & 0.000 & 0.000 & 2605 & 0.000 & 0.000 \\
		2 & 0.20 & 0.0 & 1111 & 6.295 & 0.046 & 440 & 6.332 & 0.069 \\
		3 & 0.25 & 0.0 & 1120 & 8.563 & 0.067 & 564 & 8.600 & 0.100 \\
		4 & 0.30 & 0.0 & 1703 & 9.788 & 0.079 & 1195 & 9.821 & 0.115 \\
		5 & 0.40 & 0.0 & 3197 & 10.236 & 0.090 & 2590 & 10.255 & 0.126 \\
		6 & 0.55 & 0.0 & 5483 & 8.673 & 0.097 & 4352 & 8.659 & 0.132 \\
		7 & 0.75 & 0.0 & 7825 & 4.992 & 0.101 & 6414 & 4.933 & 0.136 \\
		\rowcolor{lightgray}
		8 & 1.00 & 0.0 & 12336 & -0.728 & 0.104 & 10500 & -0.814 & 0.139 \\
		9 & 1.00 & 0.6 & 2775 & -8.337 & 0.112 & 1603 & -8.329 & 0.165 \\
		10 & 1.00 & 0.8 & 1957 & -17.853 & 0.125 & 277 & -17.632 & 0.188 \\
		11 & 1.00 & 0.9 & 2115 & -25.598 & 0.131 & 983 & -25.326 & 0.204 \\
		\rowcolor{lightgray}
		12 & 1.00 & 1.0 & 1930 & -35.722 & 0.136 & 995 & -35.443 & 0.211 \\
		\hline
	\end{tabular}
\end{table*}


\begin{table*}
	\caption{Free energy differences. Decorrelation via $\lambda$-dependent energy.}
	\label{table:glucose vdW free energies}
	\begin{tabular}{CCCCCCCCC}
		\hline
		%& \multicolumn{3}{c}{MICS} & \multicolumn{3}{c}{MBAR} \\
		\text{state} & \lambda_\text{vdW} & \lambda_\text{C} & n^\text{eff}_\text{\tiny MICS} & \Delta f_\text{\tiny MICS} &  \delta \Delta f_\text{\tiny MICS} & n_\text{\tiny MBAR} & \Delta f_\text{\tiny MBAR} & \delta \Delta f_\text{\tiny MBAR} \\
		\hline
%		1 & 0.00 & 0.0 & 2880 & 0.000 & 0.000 & 2525 & 0.000 & 0.000 \\
%		2 & 0.20 & 0.0 & 1131 & 6.294 & 0.046 & 447 & 6.405 & 0.066 \\
%		3 & 0.25 & 0.0 & 1146 & 8.562 & 0.067 & 578 & 8.676 & 0.099 \\
%		4 & 0.30 & 0.0 & 1765 & 9.787 & 0.079 & 1256 & 9.890 & 0.113 \\
%		5 & 0.40 & 0.0 & 3657 & 10.234 & 0.090 & 3054 & 10.333 & 0.123 \\
%		6 & 0.55 & 0.0 & 6970 & 8.672 & 0.097 & 5881 & 8.793 & 0.129 \\
%		7 & 0.75 & 0.0 & 5609 & 4.990 & 0.102 & 3972 & 5.138 & 0.132 \\
		8 & 1.00 & 0.0 & 3779 & -0.731 & 0.105 & 2848 & -0.538 & 0.137 \\
%		9 & 1.00 & 0.6 & 3182 & -8.362 & 0.113 & 518 & -8.246 & 0.205 \\
%		10 & 1.00 & 0.8 & 1870 & -17.881 & 0.126 & 185 & -17.917 & 0.249 \\
%		11 & 1.00 & 0.9 & 1882 & -25.626 & 0.132 & 864 & -25.776 & 0.267 \\
		12 & 1.00 & 1.0 & 1605 & -35.753 & 0.137 & 799 & -35.911 & 0.272 \\
		\hline
	\end{tabular}
\end{table*}





In order to compare the proposed method against MBAR in regard to the calculation of relative free energies of sampled states, we simulated the solvation of a glucose molecule in water. Both molecules are modeled as rigid bodies. The intermolecular interactions between water atoms and propane pseudo-atoms are calculated by the softcore potential\cite{Beutler_1994} $\nu(r,\lambda) = 4\lambda\epsilon(s^{-2} - s^{-1})$, where $s = (r/\sigma)^6 + 0.5 (1-\lambda)$.

The model used for the solvent, water, was TIP4P2005 \cite{Abascal_2005}. The solute is $\alpha$-D-glucose, modeled according to the GLYCAM force field \cite{Kirschner_2007}. Simulations were executed with LAMMPS \cite{Plimpton_1995}, with modifications.



From Ref.~\citenum{Shirts_2003_2}
\begin{equation*}
{\Delta G}_\text{solv} = RT \left[{\Delta f}_{a,b} - \ln \left(\frac{\avg{V}_b}{\avg{V}_a}\right)\right],
\end{equation*}
where subscripts $a$ and $b$ refer to the initial state (uncoupled solute) and to the final state (fully coupled solute), respectively. 

\begin{figure}
	\centering
	\includegraphics{Figures/nelder_mead}
	\caption{}
	\label{fig:nelder_mead}
\end{figure}


\subsection{Potential of Mean Force for a Host-Guest System}

WHAM method\cite{Kumar_1992} and implementation \cite{Grossfield_nodate}.

Cucurbit[7]uril. GAFF parameters \cite{Wang_2004}. AM1-BCC charges \cite{Jakalian_2000, Jakalian_2002} using the Antechamber \cite{Wang_2006} as implemented in Ambertools 2017 \cite{Case_2017}.

\section{Conclusion}

Free-energy perturbation is an accurate method when the typical configurations of the target state form a significant subset of the typical configurations of the sampled state.

We tried to interpret different methods simply as distinct choices of a reduced potential $u_0(x)$ for the reference state, expressed as an analytical function $u_0 = u_0(u_1,\dots,u_n)$.

\begin{acknowledgement}

The author acknowledges the financial support of Petrobras (project code CENPES 16113).

\end{acknowledgement}

\appendix

\section{Subsampling and Uncertainties in MBAR}
\label{sec:subsampling and uncertainty in MBAR}

For a series $\{x_{i,k}\}_{k=1}^{n_i}$ drawn at state $i$, the subsampling interval is set as the statistical inefficiency of some configurational property $A(x)$, which is computed as\cite{Chodera_2007}
\begin{equation}
\label{eq:statistical inefficiency}
g_i = 1 + 2 \sum\limits_{t=1}^{n_i-1} \frac{n_i - t}{n_i} \gamma_i(t),
\end{equation}
where $\gamma_i(t)$ is the normalized autocorrelation function of $A$, calculated by
\begin{equation*}
\gamma_i(t) = \frac{\dfrac{1}{n_i - t} \sum\limits_{k=1}^{n_i-t} \left[A(x_{i,k}) - \avg A_i\right]\left[A(x_{i,k+t}) - \avg A_i\right]}{\dfrac{1}{n_i} \sum\limits_{k=1}^{n_i} \left[A(x_{i,k}) - \avg A_i\right]^2}.
\end{equation*}

The autocorrelation function $\gamma_i(t)$ is expected to asymptotically decay to zero for most properties, but in practice its tail fluctuates around zero. Chodera \textit{et al}.\cite{Chodera_2007} proposed that the sum in Eq.~\eqref{eq:statistical inefficiency} should be truncated right before $\gamma_i(t)$ crosses zero for the first time, arguing that the function becomes statistically indistinguishable from zero at this point.

\citeauthor{Shirts_2008} \cite{Shirts_2008} established MBAR from a more general version of Eq.~\eqref{eq:mbar free energy estimator}, put forward by \citeauthor{Kong_2003} \cite{Kong_2003}, in which a function $q_i(x)$ and a constant $\hat c_i$ replace the terms $e^{-u_i(x)}$ and $e^{-\hat f_i}$, respectively. It is
\begin{equation}
\label{eq:mbar general estimator}
{\hat c}_i = \sum_{k=1}^n \frac{q_i(x_k)}{\sum_{j=1}^m n_j q_j(x_k) c_j^{-1}}.
\end{equation}

This opens up the possibility of releasing some function $q_i(x)$ from being strictly non-negative, provided that $n_i = 0$ because it would not be an actual probability density. \citeauthor{Kong_2003} \cite{Kong_2003} also developed an expression, valid for i.i.d. samples, for estimating a matrix whose each element $\hat \Theta_{i,j}$ is the asymptotic covariance between $\ln \hat c_i$ and $\ln \hat c_j$ obtained after solving the estimating equations self-consistently. Such expression is
\begin{equation}
\label{eq:mbar covariance matrix}
\hat{\mt \Theta} = \tr{\mt W} (\mt I - {\mt W}{\mt N}\tr{\mt W})^+  {\mt W},
\end{equation}
where $\mt W$ is an $n \times m$ matrix with $W_{k,i} = \frac{q_i(x_k) \hat c_i^{-1}}{\sum_{j=1}^m n_j q_j(x_k) \hat c_j^{-1}}$, $\mt I$ is the $n \times n$ identity matrix, and $\mt N = \text{diag}(\vt n)$, where $\text{diag}(\cdot)$ returns a diagonal matrix built from the entries of a vector. A Moore-Penrose pseudoinverse is required because the expression in parentheses produces a singular matrix. In fact, this can possibly be a very large matrix, but this might not become an unsurmountable issue in most cases (see Ref.~\citenum{Shirts_2008} for details). From the covariance matrix we can obtain, via the delta method, the square-error $\delta^2 h$ of any property $h$ that depends analytically on the set $\{\ln c_i\}_{i=1}^m$.

MBAR is also suitable for computing, via reweighting, ensemble averages and functions thereof (which includes relative free-energies), as well as their uncertainties, at both sampled states and unsampled ones. In a comment attached to Ref.~\citenum{Kong_2003}, Doss suggested a way of exploiting the flexibility of the $q$-functions in Eqs.~\eqref{eq:mbar general estimator} and \eqref{eq:mbar covariance matrix} in order to compute a property and its uncertainty at an unsampled state $m+1$, for which $q_{m+1}(x) = e^{u_{m+1}(x)}$. It consists in setting, in addition, a fictitious state $m+2$ for which $q_{m+2}(x) = A(x)e^{u_{m+1}(x)}$. In this way, Eq.~\eqref{eq:mbar general estimator} can be used to compute $c_{m+1}$ and $c_{m+2}$, from which one gets both $\hat f_{m+1} = -\ln c_{m+1}$ and $\bar A = {c_{m+2}}/{c_{m+1}}$. Because $n_{m+1} = n_{m+2} = 0$, such computation does not require iterations in case the $c$-constants of the sampled states have already been determined. The square-errors $\delta^2 \hat f_{m+1}$ and $\delta^2 \avg A$ then follow directly from Eq.~\eqref{eq:mbar covariance matrix} and the delta method, Eq.~\eqref{eq:delta method}.

\bibliography{mics}

\end{document}